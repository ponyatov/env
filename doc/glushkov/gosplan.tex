\documentclass{article}
% \documentclas{oneside,article}%s[oneside,10pt]{book}
\usepackage[paperwidth=118.8mm,paperheight=68.2mm,margin=2mm]{geometry}
\renewcommand{\familydefault}{\sfdefault}\normalfont
\usepackage[unicode,colorlinks=true]{hyperref}
% Cyrillization
\usepackage[utf8]{inputenc}
\usepackage[english,russian]{babel}
\usepackage{indentfirst}


\title{Как "погас" ОГАС\\Что скажет история\,?\\(отрывок из воспоминаний)}
\author{Акад. В.М. Глушков} 
\date{\copyright\ Альманах "Восток" Выпуск: № 2(14), февраль 2004 года}
\begin{document}
\maketitle

Труд, капитал, энергия
\bigskip
 
Воспроизводится по книге
Б.Н. Малиновского "История вычислительной техники в лицах"
(изд. "КИТ", ПТОО "А.С.К.", Киев, 1995, стр. 154-168).
Комментарии и вставки принадлежат Б.Н. Малиновскому.
\bigskip

Задача построения общегосударственной автоматизированной системы управления
(ОГАС) экономикой была поставлена мне первым заместителем Председателя Совета
Министров (тогда А.Н. Косыгиным) в ноябре 1962 года. К нему меня привел
президент Академии наук СССР М.В. Келдыш, с которым я поделился некоторыми
своими соображениями по этому поводу.

Когда я кратко обрисовал Косыгину, что мы хотим сделать, он одобрил наши
намерения, и вышло распоряжение Совета Министров СССР о создании специальной
комиссии под моим председательством по подготовке материалов для постановления
правительства. В эту комиссию- вошли ученые-экономисты, в частности, академик
Н.Н. Федоренко, начальник ЦСУ В.Н. Старовский, первый заместитель министра связи
А.И. Сергийчук, а также другие работники органов управления.

Комиссии и ее председателю, т.е. мне, были предоставлены определенные
полномочия. Они заключались в том, что я имел возможность прийти в любой кабинет
- к министру, председателю Госплана - и задавать вопросы или просто сесть в
уголке и смотреть, как он работает: что он решает, как решает, по каким
процедурам и т.д. Естественно, я получил разрешение ознакомиться по своему
выбору с любыми промышленными объектами - предприятиями, организациями и пр.

К этому времени у нас в стране уже имелась концепция единой системы
вычислительных центров для обработки экономической информации. Ее выдвинули
академик, виднейший экономист В.С. Немчинов и его ученики. Они предложили
использовать вычислительную технику, имевшуюся в вычислительных центрах, но не в
режиме удаленного доступа. Экономисты, да и специалисты по вычислительной
технике этого тогда не знали. Фактически они скопировали предложения,
подготовленные в 1955 году Академией наук СССР о создании системы академических
вычислительных центров для научных расчетов, в соответствии с которыми был
создан Вычислительный центр АН Украины. Они предложили сделать точно то же для
экономики: построить в Москве, Киеве, Новосибирске, Риге, Харькове и других
городах крупные вычислительные центры (государственные), которые обслуживались
бы на должном уровне и куда сотрудники различных экономических учреждений
приносили бы свои задачи, считали, получали результаты и уходили. Вот в чем
состояла их концепция. Меня, конечно, она удовлетворить не могла, так как к
этому времени мы уже управляли объектами на расстоянии, передавали данные из
глубины Атлантики прямо в Киев в вычислительный центр.

У нас в стране все организации были плохо подготовлены к восприятию обработки
экономической информации. Вина лежала как на экономистах, которые практически
ничего не считали, так и на создателях ЭВМ. В результате создалось такое
положение, что у нас органы статистики и частично плановые были снабжены
счетно-аналитическими машинами образца 1939 года, к тому времени полностью
замененными в Америке на ЭВМ.

Американцы до 1965 года развивали две линии: научных машин (это двоичные машины
с плавающей запятой, высокоразрядные) и экономических машин (последовательные
двоично-десятичные с развитой памятью и т.д.). Впервые эти две линии соединились
в машинах фирмы IBM.

У нас нечему было сливаться, так как существовали лишь машины для научных
расчетов, а машинами для экономики никто не занимался. Первое, что я тогда
сделал, - попытался заинтересовать конструкторов, в частности Б.И. Рамеева
(конструктора ЭВМ "Урал-1", "Урал-2") и В.В. Пржиялковского (конструктора ЭВМ
серии "Минск"), в необходимости разработки новых машин, ориентированных на
экономические применения.

Я организовал коллектив у нас в институте, сам разработал программу по его
ознакомлению с задачей, поставленной Косыгиным. Неделю провел в ЦСУ СССР, где
подробно изучал его работу. Просмотрел всю цепочку от районной станции до ЦСУ
СССР. Очень много времени провел в Госплане, где мне большую помощь оказали
старые его работники. Это прежде всего Василий Михайлович Рябиков, первый
заместитель председателя Госплана, ответственный за оборонную тематику, И.
Спирин, заведующий сводным сектором оборонных отраслей в Госплане СССР. У обоих
был очень большой опыт руководства военной экономикой, и, конечно, они хорошо
знали работу Госплана. С их помощью я разобрался со всеми задачами и этапами
планирования и возникающими при этом трудностями.

За 1963 год я побывал не менее чем на 100 объектах, предприятиях и организациях
самого различного профиля: от заводов и шахт до совхозов. Потом я продолжал эту
работу, и за десять лет число объектов дошло почти до тысячи. Поэтому я очень
хорошо, возможно, как никто другой, представляю себе народное хозяйство в целом:
от низа до самого верха, особенности существующей системы управления,
возникающие трудности и что надо считать. Понимание того, что нужно от техники,
у меня возникло довольно быстро. Задолго до окончания ознакомительной работы я
выдвинул концепцию не просто отдельных государственных центров, а сети
вычислительных центров с удаленным доступом, т.е. вложил в понятие коллективного
пользования современное техническое содержание.

Мы (В.М. Глушков, В.С. Михалевич, А.И. Никитин и др. - Прим. авт.) разработали
первый экскизный проект Единой Государственной сети вычислительных центров
ЕГСВЦ, который включал около 100 центров в крупных промышленных городах и
центрах экономических районов, объединенных широкополосными каналами связи. Эти
центры, распределенные по территории страны, в соответствии с конфигурацией
системы объединяются с остальными, занятыми обработкой экономической информации.
Их число мы определяли тогда в 20 тысяч. Это крупные предприятия, министерства,
а также кустовые центры, обслуживавшие мелкие предприятия. Характерным было
наличие распределенного банка данных и возможность безадресного доступа из любой
точки этой системы к любой информации после автоматической проверки полномочий
запрашивающего лица. Был разработан ряд вопросов, связанных с защитой
информации. Кроме того, в этой двухъярусной системе главные вычислительные
центры обмениваются между собой информацией не путем коммутации каналов и
коммутации сообщений, как принято сейчас, с разбивкой на письма, я предложил
соединить эти 100 или 200 центров широкополосными каналами в обход
каналообразующей аппаратуры с тем, чтобы можно было переписывать информацию с
магнитной ленты во Владивостоке на ленту в Москве без снижения скорости. Тогда
все протоколы сильно упрощаются и сеть приобретает новые свойства. Это пока
нигде в мире не реализовано. Наш проект был до 1977 года секретным.

Кроме структуры сети я сразу счел необходимым разработать систему математических
моделей для управления экономикой с тем, чтобы видеть регулярные потоки
информации. Об этом я рассказал академику В.С. Немчинову, который в то время был
тяжело болен и лежал дома, однако принял меня, выслушал и в принципе все
одобрил.

Потом я представил нашу концепцию М.В. Келдышу, который все одобрил, за
исключением безденежной системы расчетов населения, но без нее система тоже
работает. По его мнению, она вызвала бы ненужные эмоции, и вообще не следует это
смешивать с планированием. Я с ним согласился, и мы эту часть в проект не
включили. В связи с этим мной была написана отдельная записка в ЦК КПСС, которая
много раз всплывала, потом опять исчезала, но никакого решения по поводу
создания безденежной системы расчетов так и не было принято.

Закончив составление проекта, мы передали его на рассмотрение членам комиссии.

Добиваясь решения огромной по сложности и материальным затратам задачи, В.М.
Глушков в 1962 году написал статью для "Правды".

Прочитав ее, бывший научный руководитель Глушкова по докторской диссертации А.Г.
Курош, внимательно следивший за успехами талантливого ученика, написал ему:

"...Мечтая, могу представить себе Вас во главе всесоюзного органа, планирующего
и организующего перестройку управления экономикой, т.е. народным хозяйством на
базе кибернетики (в соответствии, понятно, с основными установками высших
органов страны), а также внедрение кибернетики в промышленность, науку, и, хочу
подчеркнуть, в преподавание (на всех уровнях), медицину и вообще во все виды
интеллектуальной деятельности. Было бы печально, если бы этот орган оказался
министерским или государственным комитетом, т.е. органом бюрократическим. Это
должен быть орган высокой интеллектуальности, составленный из людей, способных,
каждый в своей области, на такое же понимание больших задач, какое есть, видимо,
у Вас по проблеме в целом. Это должен быть орган почти без аппарата, орган
мыслителей, а не чиновников. Это лишь мечты, конечно, кроме вопроса о главе
этого органа, - Вы могли бы много сделать для реализации этих мечтаний..."

К сожалению, после рассмотрения проекта комиссией от него почти ничего не
осталось, вся экономическая часть была изъята, осталась только сама сеть.
Изъятые материалы уничтожались, сжигались, так как были секретными. Нам даже не
разрешали иметь копию в институте. Поэтому мы, к сожалению, не сможем их
восстановить.

Против всего проекта в целом начал резко возражать В.Н. Старовский, начальник
ЦСУ. Возражения его были демагогическими. Мы настаивали на такой новой системе
учета, чтобы из любой точки любые сведения можно было тут же получить. А он
ссылался на то, что ЦСУ было организовано по инициативе Ленина, и оно
справляется с поставленными им задачами; сумел получить от Косыгина заверения,
что той информации, которую ЦСУ дает правительству, достаточно для управления, и
поэтому ничего делать не надо.

В конце концов, когда дошло дело до утверждения проекта, все его подписали, но
при возражении ЦСУ. Так и было написано, что ЦСУ возражает против всего проекта
в целом.

В июне 1964 года мы вынесли наш проект на рассмотрение правительства. В ноябре
1964 года состоялось заседание Президиума Совета Министров, на котором я
докладывал об этом проекте. Естественно, я не умолчал о возражении ЦСУ. Решение
было такое: поручить доработку проекта ЦСУ, подключив к этому Министерство
радиопромышленности.

В течение двух лет ЦСУ сделало следующую работу. Пошли снизу, а не сверху: не от
идеи, что надо стране, а от того, что есть. Районным отделениям ЦСУ
Архангельской области и Каракалпакской АССР было поручено изучить потоки
информации - сколько документов, цифр и букв поступает в районное отделение ЦСУ
от предприятий, организаций и т.д.

По статистике ЦСУ, при обработке информации на счетно-ана\-ли\-ти\-чес\-ких
машинах на каждую вводимую цифру или букву приходится 50 сортировочных или
арифметических операций. Составители проекта с важным видом написали, что когда
будут использоваться электронные машины, операций будет в десять раз больше.
Почему это так, одному Господу Богу известно. Потом взяли количество всех
бумажек, умножили на 500 и получили производительность, требуемую от ЭВМ,
которую надо, например, установить в Архангельске и в Нукусе (в Каракалпакской
АССР). И у них получились смехотворные цифры: скорость вычислений ЭВМ должна
составлять около 2 тысяч операций в секунду или около того. И все. Вот в таком
виде подали проект в правительство.

Снова была создана комиссия по приемке, меня хотели сделать председателем, но я
отказался по этическим соображениям. С этим согласились. После ознакомления
членов комиссии с проектом возмутились представители Госплана, которые заявили,
что они не все концепции академика Глушкова разделяют, но в его проекте хотя бы
было планирование, а в этом одна статистика. Комиссия практически единогласно
отвергла этот проект, за исключением меня. Я предложил, учитывая жизненную
важность этого дела для страны, признать проект неудовлетворительным, но перейти
к разработке технического проекта, поручив это Министерству радиопромышленности,
Академии наук СССР, Госплану. С этим не согласились, мое предложение записали
как особое мнение и поручили Госплану делать заново экскизный проект.

Госплан потребовал на это два года, а был уже 1966-й. До 1968 года мусолили-
-мусолили, но абсолютно ничего не сделали. И вместо эскизного проекта
подготовили распоряжение Совета Министров СССР о том, что, поскольку очень мудро
ликвидировали совнархозы и восстановили отраслевой метод управления, то теперь
не о чем заботиться. Нужно, чтобы все министерства создали отраслевые системы, а
из них автоматически получится общегосударственная система. Все облегченно
вздохнули - ничего делать не надо, и такое распоряжение было отдано. Получился
ОГАС - сборная солянка.

В.М.Глушкова вспоминает, что не раз, возвращаясь из Москвы, муж говорил: ужасно
угнетает мысль, что никому ничего не нужно. В эти годы под стекло на столе
Глушкова в его домашнем кабинете, к ранее подсунутой записке легла еще одна:

\bigskip
"Сто раз я клятву говорил такую:

Сто лет в темнице лучше протоскую,

Сто гор скорее в ступе истолку я,

Чем истину тупице растолкую".

Бахвалан Махмуд
\bigskip

Но дело было не столько в "тупицах", сколько в сознательной дискредитации идей
ученого.

Начиная с 1964 года (времени появления моего проекта) против меня стали открыто
выступать ученые-экономисты Либерман, Белкин, Бирман и другие, многие из которых
потом уехали в США и Израиль. Косыгин, будучи очень практичным человеком,
заинтересовался возможной стоимостью нашего проекта. По предварительным
подсчетам его реализация обошлась бы в 20 миллиардов рублей. Основную часть
работы можно сделать за три пятилетки, но только при условии, что эта программа
будет организована так, как атомная и космическая. Я не скрывал от Косыгина, что
она сложнее космической и атомной программ вместе взятых и организационно
гораздо труднее, так как затрагивает все и всех: и промышленность, и торговлю,
планирующие органы, и сферу управления, и т.д. Хотя стоимость проекта
ориентировочно оценивалась в 20 миллиардов рублей, рабочая схема его реализации
предусматривала, что вложенные в первой пятилетке первые 5 миллиардов рублей в
конце пятилетки дадут отдачу более 5 миллиардов, поскольку мы предусмотрели
самоокупаемость затрат на программу. А всего за три пятилетки реализация
программы принесла бы в бюджет не менее 100 миллиардов рублей. И это еще очень
заниженная цифра.

Но наши горе-экономисты сбили Косыгина с толку тем, что) дескать, экономическая
реформа вообще ничего не будет стоить, т.е. будет стоить ровно столько, сколько
стоит бумага, на которой будет напечатано постановление Совета Министров, и даст
в результате больше. Поэтому нас отставили в сторону и, более того, стали
относиться с настороженностью. И Косыгин был недовлен. Меня вызвал Шелест и
сказал, чтобы я временно прекратил пропаганду ОГАС и занялся системами нижнего
уровня.

Вот тогда мы и начали заниматься "Львовской системой". Дмитрий Федорович Устинов
пригласил к себе руководителей оборонных министерств и дал им команду делать
все, что говорит Глушков. Причем с самого начала было предусмотрено, чтобы
системы делались для всех отраслей сразу, т.е. какой-то зачаток
общегосударственности был.

Устинов дал команду, чтобы никого из экономистов не пускали на предприятия. Мы
могли спокойно работать. И это сэкономило нам время, дало возможность
подготовить кадры. Для выполнения работы был создан ряд новых организаций -
институт Шихаева, институт Данильченко и др. - во всех отраслях по институту.
Расставили людей и начали потихоньку работать. А Институт кибернетики АН Украины
переключился в основном сначала на "Львовскую", а потом на "Кунцевскую" системы
- занялись "низом", так сказать.

Для руководства работой в оборонном комплексе был создан межведомственный
комитет (МВК) девяти отраслей под руководством министра радиопромышленности П.С.
Плешакова и совет директоров головных институтов (СДГИ) оборонных отраслей по
управлению, экономике и информатике под руководством Юрия Евгеньевича Антипова,
члена военно-промышленной комиссии ВПК. Научным руководителем комитета и совета
был В.М. Глушков. Вспоминая об этом времени, Ю.Е. Антипов пишет:

"Начиная с 1966 г. работа велась таким образом: сначала проблема, связанная с
созданием той или иной автоматизированной системы, обсуждалась на СДГИ, потом
рассматривалась на МВК, а на заседании ВПК принималось окончательное решение.

По этой схеме реализовались основные идеи, высказанные Глушковым: разработка
типовых систем для предприятий и отрасли, создание программных методов
планирования и управления, переход к системному проектированию средств передачи
и обработки информации, развитие инфраструктур информационной индустрии,
проблемы моделирования и управления и др. Я думаю, что В.М.Глушкову повезло в
том, что в "оборонке" нашлись силы для реализации его идей".

Нашлись они и на Украине. По инициативе Виктора Михайловича решением
правительства Украины в Госплане УССР был создан в 1971 г. специальный отдел с
достаточно широкими полномочиями, возглавить который был приглашен с одобрения
академика Глушкова М.Т. Матвеев. В настоящее время он директор Головного НИИ по
проблемам информатики Министерства экономики Украины, доктор экономических наук.
Практически это был опорный отдел Глушкова, который, функционируя в Госплане
УССР, стал проводником его научной политики. С такой мощной основой отделу
удалось в кратчайший срок наладить процесс планомерного внедрения компьютерных
технологий в народное хозяйство и начать проектирование и практическое
осуществление проектов РАСУ и РСВЦ на Украине. Многие годы до смерти Виктора
Михайловича Украина в СССР занимала лидирующие позиции по всей проблематике.

"Роль и заслуги Виктора Михайловича в этом трудно переоценить, - вспоминает о
том памятном времени М.Т. Матвеев. - Высокая эффективность работы всех
причастных к процессу компьютеризации обусловливалась тем, что Виктор Михайлович
любые вопросы разрешал в реальном времени, без задержек; понимание проблематики
и способность нахождения путей реализации казалось бы неразрешимых вопросов в
реальных условиях у академика были потрясающими: многонедельных и многомесячных
ожиданий аудиенций у Виктора Михайловича не практиковалось. Он активно и
результативно защищал интересы сферы компьютеризации на самом высоком
государственном уровне. Виктор Михайлович был единственным в этом плане не
только на Украине, но и в СССР. Подтверждением этому является образовавшийся и
усиливающийся застой в этой важнейшей области после его ухода. Я не могу назвать
ни одного сколь-нибудь серьезного государственного акта, принятого с тех пор,
который бы вдохнул новую жизнь в начатое им дело. Мы, его ученики и
единомышленники, хотя и старались в память о нем продвигать дальше его идеи и
замыслы, часто, очень часто ощущали невосполнимую его потерю. Глубоко убежден,
что он нашел бы выход из сложившейся сейчас совершенно нелогичной и необъяснимой
кризисной и опасной ситуации".

Действительно, в многочисленных научных и публицистических статьях и монографиях
В.М. Глушкова высказывалось и разрабатывалось множество идей по
совершенствованию системы государственного управления, в частности, созданию
более совершенных по сравнению с существующими способов регулирования
производственных и социальных процессов, пересмотру разного рода нормативов и
разработке механизмов их объективного формирования, созданию технической базы
согласования производственных программ в масштабе всей страны, обеспечению
руководителей инструментарием для формирования, моделирования и оценки
последствий принятых решений (система Дисплан. А.А. Бакаев), по использованию
более справедливых распределительных механизмов, созданию такой системы учета,
при которой выявлялись бы источники нетрудовых доходов, внедрению системы
безденежных расчетов для всего населения и пр. Многие из этих идей, казавшихся в
его время слишком революционными, сегодня приобрели новое актуальное звучание.

В конце 60-х годов в ЦК КПСС и Совете Министров СССР появилась информация о том,
что американцы еще в 1966 году сделали эскизный проект информационной сети
(точнее, нескольких сетей), т.е. на два года позже нас. В отличие от нас они не
спорили, а делали, и на 1969 год у них был запланирован пуск сети АРПАНЕТ, а
затем СЕЙБАРПАНЕТ и др., объединяющих ЭВМ, которые были установлены в различных
городах США.

Тогда забеспокоились и у нас. Я пошел к Кириленко и передал ему записку о том,
что надо возвратиться к тем идеям, которые были в моем проекте. "Напиши, что
надо делать, создадим комиссию", - сказал он. Я написал примерно такое:
"Единственное, что прошу сделать, - это не создавать по моей записке комиссию,
потому что практика показывает, что комиссия работает по принципу вычитания
умов, а не сложения, и любое дело способна загубить". Но тем не менее комиссия
была создана. Председателем назначили В.А. Кириллина (председателя ГКНТ), а меня
заместителем.

Комиссия была еще более высокого уровня - с участием министра финансов, министра
приборостроения и др. Она должна была подготовить проект решения по созданию
ОГАС. И мы должны были вынести эти материалы на рассмотрение Политбюро ЦК КПСС,
а Политбюро уже решало, что пойдет на съезд.

Раоота началась. И тут я основное внимание уделил уже не столько сути дела,
поскольку в проекте она содержалась, сколько механизму реализации ОГАС.

Дело в том, что у Королева или Курчатова был шеф со стороны Политбюро, и они
могли прийти к нему и сразу решить любой вопрос. Наша беда была в том, что по
нашей работе такое лицо отсутствовало. А вопросы были здесь более сложные,
потому что затрагивали политику, и любая ошибка могла иметь трагические
последствия. Поэтому тем более была важна связь с кем-то из членов Политбюро,
потому что это задача не только научно-техническая, но прежде всего
политическая.

Мы предусматривали создание Государственного комитета по совершенствованию
управления (Госкомупра), научного центра при нем в составе 10--15 институтов,
причем институты уже почти все существовали в то время\ --- нужно было создать
заново только один, головной. Остальные можно было забрать из отраслей или
Академии наук или частично переподчинить. И должен быть ответственный за все это
дело от Политбюро.

Все шло гладко, все соглашались. В это время уже был опубликован проект
директив XXVI съезда, включавший все наши формулировки, подготовленные на
комиссии.

На Полибюро дважды рассматривался наш вопрос. На одном заседании была
рассмотрена суть дела, с ней согласились и сказали, что ОГАС надо делать. А вот
как делать - Госкомупр-ли или что-то другое, - это вызвало споры.

Мне удалось "додавить" всех членов комиссии, один Гарбузов не подписал наши
предложения. Но мы все-таки внесли их на Политбюро.

А когда мы пришли на заседание (а оно, кстати, проходило в бывшем кабинете
Сталина), то Кириллин мне шепнул: что-то, мол, произошло, но что\ --- он не
знает. Вопрос рассматривался на заседании, без Генерального секретаря (Брежнев
уехал в Баку на празднование 50-летия советской власти в Азербайджане), Косыгина
(он был в Египте на похоронах А.Насера). Вел заседание Суслов. Вначале
предоставили слово Кириллину, потом мне. Я выступил коротко, но вопросов было
задано очень много. Я ответил на все. Потом были приглашены заместители
Косыгина, выступил Байбаков. Он сказал так:

"Смирнов поддержал, и, в общем, все зампреды поддержали наши предложения. Я
слышал, что здесь есть возражения у товарища Гарбузова. (министра финансов. -
Прим. авт.). Если они касаются увеличения аппарата, то я считаю дело настолько
важным, что если Политбюро только в этом усматривает трудность, то пусть мне
дадут поручение, как председателю Госплана, и я внесу предложение о ликвидации
трех министерств (сократить или объединить) и тогда найдется штат для этого
дела".

К.Б. Руднев (министр ПСА и СУ. - Прим. авт.) откололся. Он, хотя и подписал наш
документ, но здесь выступил и сказал, что это, может, преждевременно - как-то
так.

Гарбузов выступил так, что сказанное им годится для анекдота. Вышел на трибуну и
обращается к Мазурову (он тогда был первым заместителем Косыгина). Вот, мол,
Кирилл Трофимович, по вашему поручению я ездил в Минск, и мы осматривали
птицеводческие фермы. И там на такой-то птицеводческой ферме (назвал ее)
птичницы сами разработали вычислительную машину.

Тут я громко засмеялся. Он мне погрозил пальцем и сказал: "Вы, Глушков, не
смейтесь, здесь о серьезных вещах говорят" Но его Суслов перебил: "Товарищ
Гарбузов, вы пока еще тут не председатель, и не ваше дело наводить порядок на
заседании Политбюро". А он - как ни в чем не бывало, такой самоуверенный и
самовлюбленный человек, продолжает: "Три программы выполняет: включает музыку,
когда курица снесла яйцо, свет выключает и зажигает и все такое прочее. На ферме
яйценосность повысилась". Вот, говорит, что нам надо делать: сначала все
птицефермы в Советском Союзе автоматизировать, а потом уже думать про всякие
глупости вроде общегосударственной системы. (А я, правда, здесь засмеялся, а не
тогда.) Ладно, не в этом дело.

Было вынесено контрпредложение, которое все снижало на порядок: вместо
Госкомупра - Главное Управление по вычислительной технике при ГКНТ, вместо
научного центра - ВНИИПОУ и т.д. А задача оставалась прежней, но она
техницизировалась, т.е. изменялась в сторону Государственной сети вычислительных
центров, а что касалось экономики, разработки математических моделей для ОГАС и
т.д. - все это смазали.

Под конец выступает Суслов и говорит: "Товарищи, может быть, мы совершаем сейчас
ошибку, не принимая проект в полной мере, но это настолько революционное
преображение, что нам трудно сейчас его осуществить. Давайте пока попробуем вот
так, а потом будет видно, как быть" И спрашивает не Кириллина, а меня: "Как вы
думаете?". А я говорю: "Михаил Андреевич, я могу вам только одно сказать: если
мы сейчас этого не сделаем, то во второй половине 70-х годов советская экономика
столкнется с такими трудностями, что все равно к этому вопросу придется
вернуться". Но с моим мнением не посчитались, приняли контрпредложение.

Ну, и работа закрутилась. Да, а тогда, когда создавалась моя первая комиссия в
1962 году, то одновременно в ГКНТ было создано Главное управление по
вычислительной технике. Оно проработало два с лишним года, а потом, когда
восстановили министерства и образовалось министерство Руднева, то управление в
1966 году ликвидировали и Руднев забрал оттуда людей к себе в Министерство
приборостроения и средств автоматизации. А теперь его воссоздали заново.

Где-то в ноябре меня приглашает Кириленко. Я пришел в его приемную на Старой
площади без двух минут десять. Там сидел наш ракетный министр С.А. Афанасьев,
которого вызвали на 10.10. Спрашивает меня: "У вас короткий вопрос?" А я ему
отвечаю, что не знаю, зачем меня позвали.

Захожу первым. Встает Андрей Павлович, поздравляет и говорит:

"Назначаешься первым заместителем Кириллина (на то место, которое занимает
сейчас Д.Г. Жимерин). Я уже согласовал это с Леонидом Ильичем, он спросил -
может, ему поговорить с тобой, но я ответил - не надо, я сам все улажу".

"Андрей Павлович, - отвечаю я ему, - а вы со мной предварительно поговорили на
эту тему? А может, я не согласен? Вы же знаете, что я возражал, я считаю, что, в
том виде, как оно сейчас принято, решение способно только исказить идею, ничего
из этого не получится. И если я приму ваше предложение, то виноваты будем мы с
вами: я внес предложение, вы поддержали, меня назначили, дали, вроде, в руки
все, - а ничего нет. Вы - умный человек, понимаете, что с таких позиций даже
простую ракету сделать нельзя, не то что построить сделать нельзя, не то что
построить новую экономическую систему управления государством".

Сели мы, и начал он меня уговаривать. Мол, вы меня ставите в неудобное положение
перед Леонидом Ильичем, я ему сказал, что все улажено. А я не поддаюсь. Тогда он
перешел на крепкие слова и выражения, а я - все равно. Потом опять на мягкие,
опять на крепкие. В общем, в час с лишним он меня отпустил. Так мы ни о чем и не
договорились. Он даже не попрощался со мной, и мы до XXIV съезда с ним, когда
встречались, не здоровались и не разговаривали.

Позднее отношения восстановились. А тогда он своего друга Жимерина предложил
заместителем Кириллина. А я согласился быть научным руководителем ВНИИПОУ.

Тем временем началась вакханалия в западной прессе. Вначале фактически никто
ничего не знал о наших предложениях, они были секретными. Первый документ,
который появился в печати, - это был проект директив XXIV съезда, где было
написано об ОГАС, ГСВЦ и т.д.

Первыми заволновались американцы. Они, конечно, не на войну с нами делают ставку
- это только прикрытие, они стремятся гонкой вооружений задавить нашу экономику,
и без того слабую. И, конечно, любое укрепление нашей экономики - это для них
самое страшное из всего, что только может быть. Поэтому они сразу открыли огонь
по мне из всех возможных калибров. Появились сначала две статьи: одна в
"Вашингтон пост" Виктора Зорзы, а другая - в английской "Гардиан". Первая
называлась "Перфокарта управляет Кремлем" и была рассчитана на наших
руководителей. Там было написано следующее: "Царь советской кибернетики академик
В.М.Глушков предлагает заменить кремлевских руководителей вычислительными
машинами". Ну и так далее, низкопробная статья.

Статья в "Гардиан" была рассчитана на советскую интеллигенцию. Там было сказано,
что академик Глушков предлагает создать сеть вычислительных центров с банками
данных, что это звучит очень современно, и это более передовое, чем есть сейчас
на Западе, но делается не для экономики, а на самом деле это заказ КГБ,
направленный на то, чтобы упрятать мысли советских граждан в банки данных и
следить за каждым человеком.

Эту вторую статью все "голоса" передавали раз пятнадцать на разных языках на
Советский Союз и страны социалистического лагеря.

Потом последовала целая серия перепечаток этих грязных пасквилей в других
ведущих капиталистических газетах - и американских, и западноевропейских, и
серия новых статей. Тогда же начали случаться странные вещи. В 1970 году я летел
из Монреаля в Москву самолетом Ил-62. Опытный летчик почувствовал что-то
неладное, когда мы летели уже над Атлантикой, и возвратился назад. Оказалось,
что в горючее что-то подсыпали. Слава Богу, все обошлось, но так и осталось
загадкой, кто и зачем это сделал. А немного позже в Югославии на нашу машину
чуть не налетел грузовик, - наш шофер чудом сумел увернуться.

И вся наша оппозиция, в частности экономическая, на меня ополчилась. В начале
1972 года в "Известиях" была опубликована статья "Уроки электронного бума",
написанная Мильнером, заместителем Г.А. Арбатова - директора Института
Соединенных Штатов Америки. В ней он пытался доказать, что в США спрос на
вычислительные машины упал. В ряде докладных записок в ЦК КПСС от экономистов,
побывавших в командировках в США, использование вычислительной техники для
управления экономикой приравнивалось к моде на абстрактную живопись. Мол
капиталисты покупают машины только потому, что это модно, дабы не показаться
несовременными. Это все дезориентировало руководство.

Да, я забыл сказать, что еще способствовало отрицательному решению по нашему
предложению. Дело в том, что Гарбузов сказал Косыгину, что Госкомупр станет
организацией, с помощью которой ЦК КПСС будет контролировать, правильно ли
Косыгин и Совет Министров в целом управляют экономикой. И этим настроил Косыгина
против нас, а раз он возражал, то, естественно, предложение о Госкомупре не
могло быть принято. Но это стало известно мне года через два.

А дальше была предпринята кампания на переориентацию основных усилий и средств
на управление технологическими процессами. Этот удар был очень точно рассчитан,
потому что и Кириленко, и Леонид Ильич - технологи по образованию, поэтому это
им было близко и понятно.

В 1972 году состоялось Всесоюзное совещание под руководством А.П. Кириленко, на
котором главный крен был сделан в сторону управления технологическими процессами
с целью замедлить работы по АСУ, а АСУ ТП дать полный ход.

Отчеты, которые направлялись в ЦК КПСС, явились, по-моему, умело организованной
американским ЦРУ кампанией дезинформации против попыток улучшения нашей
экономики. Они правильно рассчитали, что такая диверсия - наиболее простой
способ выиграть экономическое соревнование, дешевый и верный. Мне удалось
кое-что сделать, чтобы противодействовать этому. Я попросил нашего советника по
науке в Вашингтоне составить доклад о том, как "упала" популярность машин в США
на самом деле, который бывший посол Добрынин прислал в ЦК КПСС. Такие доклады,
особенно посла ведущей державы, рассылались всем членам Политбюро и те их
читали. Расчет оказался верным, и это немного смягчило удар. Так что полностью
ликвидировать тематику по АСУ не удалось.

"ОГАС погас!" - злословили враги ученого и в СССР и за рубежом. И все-таки
старания Глушкова не пропали даром. Косыгин как-то спросил его: а можно ли
увидеть что-нибудь из того, о чем вы постоянно говорите? Глушков порекомендовал
ознакомиться с тем, что сделано в оборонной промышленности, в частности, в
институте, руководимом И.А. Данильченко, который был тогда главным конструктором
по АСУ и внедрению вычислительной техники в оборонную промышленность. Глушков
был научным руководителем этих работ и был уверен, что они произведут на
Косыгина большое впечатление.

О том, что председатель Совета Министров собирается посетить институт,
Данильченко узнал от министра оборонной промышленности С.А. Зверева,
позвонившего ему накануне визита. В это время Глушкова в Москве не оыло. И хотя
Данильченко считал, что высоких гостей должен принимать научный руководитель, он
уже не смог ничего сделать. Пришлось ограничиться разговором с Глушковым по
телефону.

В десять часов утра в институт приехал Косыгин, министр обороны Устинов,
министры основных отраслей промышленности. (Далее я рассказываю со слов
Данильченко).

Визит длился день - до одиннадцати часов ночи.

Данильченко рассказал гостям о типовой АСУ для оборонных предприятий, о только
что созданной сети передачи данных, об использовании вычислительной техники на
предприятиях оборонных отраслей. Все шло "гладко", чувствовалось, что посетители
довольны увиденным и услышанным.

Когда визит близился к концу (было девять часов вечера) и, казалось, что он
благополучно закончится, Косыгин неожиданно сказал:

- По имеющимся сведениям, в одной из ведущих западных стран подготовлен доклад о
производстве и применении вычислительной техники в СССР. Там сказано, что машин
у нас меньше и они хуже и в то же время недоиспользуются. Почему это происходит?
И правильно ли это?

Данильченко понимал, как много зависит от того, что он скажет, и, пытаясь
собраться с мыслями, вспомнил совет Глушкова: в любых ситуациях говорить только
правду!

- Да! Все это верно! - ответил он.

- Причины? - резко спросил Косыгин.

- Не соблюдается основной принцип руководителя, выдвинутый академиком Глушковым,
- принцип первого лица! Руководители страны психологически не воспринимают ЭВМ,
и это самым отрицательным образом влияет на развитие и использование
вычислительной техники в стране!

Косыгин внимательно слушал, остальные молчали, поглядывая то на председателя
Совета Министров, то на ответчика.

Данильченко - по званию он был генералом, - словно рапортуя, продолжал:

- Главная задача - преодолеть психологический барьер в высшей сфере руководства.
Иначе ни Глушков, ни я, никто другой ничего не сделает. Надо обучить верхние
эшелоны власти вычислительной технике, показать ее возможности, повернуть
руководителей лицом к новой технике. Академик Глушков писал об этом в ЦК КПСС и
Совет Министров СССР, но безрезультатно. Он просил меня сказать об этом!

А.Н. Косыгин спокойно выслушал глубоко взволнованного Данильченко и, не подводя
никаких итогов, попрощался и уехал, захватив с собой министра оборонной
промышленности Зверева.

Остальные решили подождать каких-либо известий о реакции Косыгина. В половине
двенадцатого ночи позвонил Зверев и попросил к телефону Устинова.

- Косыгин очень доволен встречей, - сказал он, - теперь будут большие перемены!

И они действительно начались. Вначале была организована, специальная школа,
преобразованная через три месяца в Институт управления народным хозяйством. В
первом составе слушателей были союзные министры, во втором - министры союзных
республик, после них - их заместители и другие ответственные лица. Лекции на
первом потоке открыл Косыгин. Он же присутствовал на выпуске слушателей школы,
которым, кстати сказать, пришлось сдавать настоящие экзамены.

Лекции читались Глушковым, другими ведущими учеными страны. - И дело пошло!
Принцип "первого лица" Глушкова сработал! Министры, разобравшись, в чем дело,
сами стали проявлять инициативу. Многое было сделано. Но когда Косыгина не
стало, "принцип первого лица" снова сработал, на этот раз в обратную сторону.

Во время подготовки XXV съезда КПСС была предпринята попытка вообще изъять слово
"ОГАС" из проекта решения. Я написал записку в ЦК КПСС, когда был уже
опубликован проект "Основных направлений", и предложил создавать отраслевые
системы управления с последующим объединением их в ОГАС. И это было принято.

При подготовке XXVI съезда было то же самое. Но мы лучше подготовились:
передали материалы в комиссию, которая составляла речь Брежнева (отчетный
доклад). Я заинтересовал почти всех членов комиссии, самый главный из тех кто
готовил речь, - Цуканов - съездил в институт к Данильченко, после чего он обещал
наши предложения проталкивать. Вначале хотели их включить в речь Брежнева на
Октябрьском (1980) пленуме ЦК КПСС, потом пытались включить в отчетный доклад,
но он оказался и так слишком длинным, пришлось многое выкинуть. Тем не менее в
отчетном докладе про вычислительную технику было сказано больше, чем хотели
вначале.

Мне посоветовали развернуть кампанию за создание ОГАС в "Правде". Редактор этой
газеты, бывший управленец, меня поддержал. И то, что моей статье дали заголовок
"Дело всей страны" (статья в "Правде" называлась "Для всей страны". - Прим.
авт.), вряд ли было случайностью. "Правда" - орган ЦК КПСС, значит, статью там
обсуждали и одобрили.

Рассказ об ОГАС был записан дочерью Ольгой 11 января 1982 года.

После статьи в газете "Правда" у ученого появилась надежда, что ОГАС, наконец,
станет делом всей страны. Не это-ли заставило тяжелобольного человека держаться
и диктовать последние строки?

В этот день к нему в реанимационную палату пришел помощник министра обороны СССР
Устинова и спросил - не может ли министр чем-либо помочь? Ученый, только что
закончивший рассказ о своем "хождении по мукам", не мог не помнить о той стене
бюрократии и непонимания, которую так и не сумел протаранить, пытаясь "пробить"
ОГАС. "Пусть пришлет танк!" - гневно ответил он, обложенный трубками и проводами
от приборов, поддерживающих едва теплящуюся жизнь. Мозг его был ясен и в эти
тяжелые минуты, но терпению переносить душевные и физические муки уже приходил
конец...

История подтвердила, что слова В.М. Глушкова о том, что советская экономика в
конце 70-х годов столкнется с огромными трудностями, оказались пророческими.

До конца жизни он оставался верным своей идее создания ОГАС, реализация которой
могла бы спасти хиреющую экономику. Может, он был безнадежным мечтателем?
Ученым-романтиком? История еще скажет свое последнее слово. Отметим лишь, что
"отрицатели" его идей на Западе пошли его путем и сейчас не стесняются ссылаться
на то, что осуществляют его замыслы. Выходит, прав был ученый, говоря о причинах
обрушившейся на него критики в зарубежных средствах информации!

Его рассказ о борьбе за создание ОГАС - это обвинительный акт в адрес
руководителей государства, не сумевшим в полной мере использовать могучий талант
ученого. Если бы только Глушкова! Нет сомнения, что это одна из важных причин,
по которым великая страна споткнулась на пороге XXI века, надолго лишив миллионы
людей уверенности в завтрашнем дне, в достойном будущем своих детей, веры в то,
что они жили, живут и будут жить не зря.

Наличие планового хозяйства в бывшем СССР позволило создать самую эффективную
систему управления экономикой. Понимая это, В.М. Глушков и сделал ставку на
ОГАС. По оценке специалистов, существовавшая в СССР система управления была
втрое дешевле американской, когда США имели такой же валовой национальный
продукт. Неприятие ОГАС было стратегической ошибкой нашего руководства, нашего
общества, так как создание ОГАС давало уникальную возможность объединить
информационную и телекоммуникационную структуру в стране в единую систему,
позволявшую на новом научно-техническом уровне решать вопросы экономики,
образования, здравоохранения, экологии, сделать доступными для всех интегральные
банки данных и знаний по основным проблемам науки и техники, интегрироваться в
международную информационную систему.

Реализацию ОГАС в годы жизни В.М. Глушкова могла бы вывести страну на новый
уровень развития, соответствующий постиндустриальному обществу.

Помешали созданию ОГАС "некомпетентность высшего звена руководства, нежелание
среднего бюрократического звена работать под жестким контролем и на основе
объективной информации, собираемой и обрабатываемой с помощью ЭВМ, неготовность
общества в целом, несовершенство существовавших в то время технических средств,
непонимание, а то и противодействие ученых экономистов новым методам
управления". (Из письма, полученного автором от Ю.Е. Антипова.)

Можно соглашаться и не соглашаться с одним из ярких представителей
командно-административной системы, сторонника Глушкова в борьбе за ОГАС, но ясно
одно: Глушков был безусловно прав, ставя задачу информатизации и компьютеризации
страны. Но в тех условиях он не мог что-либо сделать без крупномасштабного
решения правительства и ЦК КПСС, которое и стало барьером на его пути. Ясно и
то, что ученый опередил время: государство и общество не были готовы к
восприятию ОГАС. Это обернулось трагедией для ученого, не желавшего смириться с
непониманием того, что для него было абсолютно очевидным.

Утром 30 января на глазах у находившихся в палате И.А. Данильченко и Ю.А.
Михеева голубые всплески на экране монитора, фиксировавшего работу сердца, вдруг
исчезли, их сменила прямая линия, - сердце ученого перестало биться...

Для заключительной оценки личности В.М. Глушкова лучше всего подходят слова
президента Национальной академии наук Украины Б.Е. Патона:

"В.М. Глушков - блестящий, истинно выдающийся ученый современности, внесший
огромный вклад в становление кибернетики и вычислительной техники в Украине и
бывшем Советском Союзе, да и в мире в целом.

своими работами предвосхитил многое из того, что сейчас появилось в
информатизированном западном обществе.

Виктор Михайлович обладал огромными разносторонними знаниями, а его эрудиция
просто поражала всех с ним соприкасавшихся. Вечный поиск нового, стремление к
прогрессу в науке, технике, обществе были замечательными его чертами.

В.М. Глушков был подлинным подвижником в науке, обладавшим гигантской
работоспособностью и трудолюбием. Он щедро делился своими знаниями, идеями,
опытом с окружающими его людьми.

В.М. Глушков внес большой вклад в развитие АН Украины, будучи с 1962 года ее
вице-президентом. Он существенно влиял на развитие научных направлений,
связанных с естественными и техническими науками. Велик его вклад в
компьютеризацию и информатизацию науки, техники, общества.

Виктора Михайловича смело можно отнести к государственным деятелям, отдававшим
всего себя служению Отечеству, своему народу. Его знали и уважали люди во всех
уголках Советского Союза. Он не жалел сил для пропаганды достижений науки,
научно-технического прогресса, общался с учеными многих зарубежных стран. Его
работы и достижения руководимого им Института кибернетики АН Украины были хорошо
известны за рубежом, где он пользовался заслуженным авторитетом.

Xopoшo понимая значение укрепления обороноспособности своей страны, В.М. Глушков
вместе с руководимым им институтом выполнил большой комплекс работ оборонного
значения. И здесь он всегда вносил свое, новое, преодолевая многочисленные
трудности, а иногда и простое непонимание. Он действительно болел за страну, ей
и науке отдал свою замечательную жизнь.

\end{document}