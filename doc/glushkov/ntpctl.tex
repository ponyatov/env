\documentclass{article}
% \documentclas{oneside,article}%s[oneside,10pt]{book}
\usepackage[paperwidth=118.8mm,paperheight=68.2mm,margin=2mm]{geometry}
\renewcommand{\familydefault}{\sfdefault}\normalfont
\usepackage[unicode,colorlinks=true]{hyperref}
% Cyrillization
\usepackage[utf8]{inputenc}
\usepackage[english,russian]{babel}
\usepackage{indentfirst}

\title{УПРАВЛЕНИЕ\\НАУЧНО-ТЕХНИЧЕСКИМ\\ПРОГРЕССОМ}
\author{концепция В. М. Глушкова}
\begin{document}
\maketitle\clearpage
\tableofcontents\clearpage

Сфера работы ОГАС РАСУ ОАСУ АСПР
 
\href{http://commons.com.ua/upravlenie-nauchno-tehnicheskim-progressom-kontseptsiya-v-m-glushkova/}{Источник:
Глушкова Віра, грудень 22, 2016}

Статья подготовлена в рамках работы Центра социальных и трудовых исследований

\bigskip

В нынешнее время бурного развития информационно-компьютерных технологий особое
значение приобретает не только прогнозирование достижения научных результатов,
но и управление научно-техническим прогрессом (НТП). Однако в условиях рыночной
экономики сделать это весьма затруднительно. Если говорить об управлении научным
прогрессом, то сегодня такое управление возможно разве что в рамках крупных
научных проектов. Всеобъемлющее же управление НТП в отдельной стране и, тем
более, в международном масштабе возможно лишь при смене нынешнего экономического
уклада. Тем не менее, разработка и актуализация перспективных схем управления
научным прогрессом необходимы для продвижения такой альтернативы.
 
Рассмотрим в нашей статье наработки киевской школы кибернетики, оценим
перспективность их применения в нынешних условиях.
 
\section{Концепция управления НТП:\\параллельность вместо последовательности}
 
Академик В. М. Глушков был пионером в разработке концепции управления НТП.
Результаты двадцатилетней работы в этом направлении Киевского Института
кибернетики были представлены Глушковым в 1980 году в работе «Управление
научно-техническим прогрессом» \cite{b1}.
 
Одной из основных идей по управлению НТП было то, что проблема должна решаться
не последовательным путем, а параллельным (с одновременным выполнением различных
задач проекта, необходимых для его реализации).
 
Конечно, последовательный метод проще, однако, цена этого подхода – резкое
падение темпов НТП. При последовательном способе управления – каждая отрасль
вынуждена ждать пока соседи, от которых они зависят, не полностью окончат свои
очередные шаги. Однако многие работы в этих последовательных цепочках могут быть
запараллелены.
 
Опыт такого запараллеливания в СССР был. Это и атомная, и космические программы.
Подобный опыт, хотя и в меньшем масштабе, был использован Глушковым и его
коллегами при разработке и внедрении первой отечественной управляющей ЭВМ Днепр.
От идеи до начала выпуска машины крупной серией прошло менее трех лет.
Подготовка производства (перестройка завода и переобучение персонала) началась
еще до полного окончания научно-исследовательских работ. Опытный образец делался
уже вместе с заводом. Уже на стадии научных разработок началась подготовка
будущих пользователей. С помощью стационарной ЭВМ «Киев» проводились удаленные
эксперименты по управлению различными технологическими процессами (в
металлургии, химии, машиностроении). В процессе этих экспериментов накапливался
опыт, готовилось математическое обеспечение, самое главное, – учились люди.
 
Но такой переход на параллельные методы управления НТП увеличивает нагрузку на
органы управления в десятки, сотни, а порою и во многие тысячи раз. Из этого
В.М. Глушков делает вывод, что решение проблемы в масштабах страны невозможно
без коренной перестройки традиционной технологии планирования и управления, а
также перехода к безбумажной информатике.
 
\section{Безбумажная информатика\\и второй информационный барьер}
 
В развитии человеческого общества неизбежно наступает момент, когда резервы
традиционных приемов совершенствования управления— организация и
социально-экономические механизмы — оказываются исчерпанными. Ведь пропускная
способность человеческого мозга, как преобразователя информации, хотя и велика,
но ограничена. В своей книге «Основы безбумажной информатики» \cite{b4} Глушков
называет такую ситуацию – вторым информационным барьером (Первым информационным
барьером называется порог сложности управления системой, превосходящей
возможности одного человека). Для преодоления второго информационного барьера
человек должен большинство функций по преобразованию информации и управлению
передать компьютеру.
 
Безбумажная информатика, в частности, предполагает переход к электронному
документообороту. К сожалению, в Украине даже эта задача полностью не решена.
 
Глушков писал: «Конечная цель становления новой технологии управления НТП
состоит в соединении в единую систему (связанную безбумажными информационными
потоками) рабочих мест всех тех, кто этот прогресс определяет и организует: от
отдельных ученых, конструкторов, проектантов и технологов – до Госплана СССР и
Госкомитета по науке и технике… Важно подчеркнуть, что речь идет не о формальном
соединении рабочих мест, а о принципиально новой технологии планирования и
управления, включающей не только перестройку технической базы, но и методологии,
организационных форм, показателей и систем стимулирования.
 
В эскизном варианте новая технология планирования и управления была проработана
еще в 60-е гг. В 1970-х годах многое было сделано и в смысле практической
реализации ее отдельных звеньев, в первую очередь низовых. Институтом
кибернетики АН УССР созданы и внедрены системы комплексной автоматизации
проектно-конструкторского труда на основе безбумажной информатики. Они позволяют
существенно (от 2--3 до 20--30 раз) сократить сроки проектирования и резко
повысить качество проектов.
 
Созданы и испытаны многие системы моделей и программ для более высоких звеньев
управления НТП. Однако их внедрение тормозится из-за того, что нынешняя
методология и организация управления не приспособлены к этим моделям и не
заинтересованы в их использовании. Прежде всего, это относится к
программно-целевому управлению».\cite{b1}
 
Ситуация, о которой писал В. М. Глушков, очень перекликается с нынешней. Так,
сегодня большинство государственных проектов по информационным технологиям в
Украине не выполняется и не внедряется из-за отсутствия заинтересованности, или
из-за наличия междуведомственных противоречий.
 
\section{Программно-целевое управление}
 
Что же нужно было сделать, по мнению кибернетиков, для того, чтобы
программно-целевое управление стало тем, чем оно действительно может и должно
быть?
 
Прежде всего, это масштабность цели. Государственная комплексная целевая
программа должна решать не только узковедомственные задачи уровня «создания
нового типа автомобиля», а задачи гораздо крупнее, например, «вывод на
определенный технико-научный уровень пассажирского транспорта». В такой
программе должны решаться не только вопросы создании новых типов автомобилей, но
и вопросы их обслуживания и эксплуатации (дороги, гаражи, организация
профилактики и ремонтов и др.), а также комплекс социально-экономических
вопросов (соотношение между общественным и личным транспортом и т.д.).

При этом очень важно, чтобы были точно определены социально-экономические
результаты, которые общество получит в результате решения этой программы.
Решение подобных задач возможно лишь при тесной увязке процесса формирования
программ с процессом долгосрочного планирования. Для успеха любой программы
чрезвычайно важно организовать правильное управление процессом ее формирования и
выполнения. Прежде всего, необходимо ввести персональную ответственность за
программу, с соответствующими правами и возможностями по распределению
материальных ресурсов для выполнения планов. На каждом из следующих уровней
планирования осуществляется вариантная агрегация планов 1 на более крупные
участки с обязательным условием сохранения персональной ответственности.

Необходимо создавать центры управления программами. Модели для такого управления
составляются на основе сетевых графиков. Потом происходит запараллеливание этих
графиков на разных этапах. Большая часть работ или почти вся работа должна
проходить в автоматическом режиме. Нормативы трудовых и материальных затрат
также, как правило, не разрабатываются полностью заново, а появляются в
результате уточнения прототипов. Технические нормативы на изготовление новых
конструкций должны прописываться уже на этапе технического задания.

Сетевые графики и нормативы должны постоянно корректироваться. Важно то, что
схема управления на основе сетевых графиков не сводится только к задаче
отыскания критического пути. Речь идет о непрерывно действующей системе
оптимизации работ, на основе постоянно уточняемой информации. Это позволяет
планировать и осуществлять разные корректирующие мероприятия.

Именно при таком подходе определяются и ранжируются возможные будущие дефициты
ресурсов и приоритеты мероприятий по их экономии.
 
\section{Нормативно-целевое прогнозирование}
 
Еще одно замечание касается степени определенности различных этапов программы.
Ведь в реальной жизни в каждый данный момент программы может находиться в разных
стадиях своего выполнения. Если на стадии массового внедрения строительства
объектов по уже законченным проектам можно и должно иметь устоявшуюся схему
предстоящих работ с достаточно выверенными нормативами, то на стадии
научно-исследовательских работ говорить об этом, как правило, преждевременно.
Вместо четкого нормативного сетевого графика предстоящих работ в этом случае
нужно иметь дело с нормативным целевым прогнозом. Методика такого
прогнозирования, предложенная Глушковым в 1969 г. \cite{b2}, и прошла успешную
апробацию в рамках СЭВ 2. \cite{b3}
 
Методика специально приспособлена для работы в описываемой системе управления
НТП на самых разных стадиях формирования комплексных целевых программ. Прежде
всего, уточняется формулировка цели, которую предполагается достигнуть с помощью
формируемой программы.
 
Например: «обеспечить дополнительное годовое производство одного триллиона
киловатт-часов электроэнергии без увеличения затрат угля, нефти, газа». На этой
стадии лимиты на затраты ресурсов для реализации программы пока не фиксируются
даже ориентировочно. Речь идет лишь о прогнозе различных вариантов сроков и
путей достижения поставленной цели, а также затрат ресурсов по этим вариантам.
 
Смысл методики заключается в последовательном (от конечной цели) разворачивании
дерева подцелей. С этой целью В. М. Глушковым была разработана специальная
методика прогнозирования на основании метода экспертных оценок \cite{b2},
которая позволяет оптимально агрегировать мнения экспертов (противоречивые, а
порой и прямо противоположные).
 
\section{Метод прогнозного графа}
 
При использовании такого метода «прогнозного дерева» обеспечивается возможность
формирования не одного, а множества различных вариантов научно-технического
развития. Последующий анализ модели позволяет определять оптимальные пути
достижения целей. При таком подходе к разработке прогнозов повышается
обоснованность решений, принимаемых в области планирования и управления
процессами научно-технического и экономического развития.
 
Еще одна особенность методики Глушкова была в том, что метод «прогнозного
дерева» встраивался в ОГАС. Поэтому в «прогнозном дереве» были предусмотрены
дополнительные возможности, а именно: непрерывное уточнение оценок экспертов, а
также работа целой системы научно-информационной службы, которая бы вовремя
оповещала группы экспертов о новых разработках, патентах, статьях по той или
иной тематике. Результаты работы такого метода (планы и прогнозы) должны были
передаваться в различные подсистемы ОГАС, например, в ДИСПЛАН (диалоговую
систему планирования) [7], в качестве возможных планов. Прогноз переводится в
план, когда сделан выбор по всем предоставляемым им альтернативам. В дальнейшем
эти планы должны были уточняться и корректироваться в режиме on-line.
 
В результате применения метода получаем вероятностные оценки сроков и путей
достижения поставленных целей. При изменении мнения экспертов прогноз в режиме
on-line пересчитывается. Подобная динамичность является обязательным требованием
любого научно-технического прогноза. Без этого в силу непрерывности развития
науки и научно-технических возможностей прогноз быстро устаревает и не только не
помогает, а подчас и вредит делу.
 
Методика предусматривает управление прогнозом на основе постоянной работы с
экспертами, постановки дополнительных научно-исследовательских работ и других
мер. Цель такого управления состоит в том, чтобы последовательно уточнять дерево
прогноза, особенно в его близких по времени частях, своевременно отбрасывать
бесперспективные варианты, получая, в конечном счете, на период 5--10 лет уже не
прогнозный граф, а сетевой график соответствующей программы. При этом попутно
решается задача определения соисполнителей и точная формулировка заданий на их
работу. Это такая важная особенность методики, позволяющая свести к минимуму
усилия по разработке схем управления программами.
 
Хочется отметить, что метод «прогнозного графа» или «прогнозного дерева»
Глушкова, как отдельный метод, и сегодня может с успехом применяться в 
современных экспертных системах для решения всевозможных задач с высокой
степенью неопределенности. Таковыми являются задачи определения путей и
результатов развития научно-технического прогресса, а также задачи
прогнозирования и планирования сложных социальных и экономических процессов.
 
Как уже отмечалось выше, схемы управления охватывают все стадии жизненных циклов
программы, включая этап строительства и реконструкции предприятий и организаций
массового производства на новой технической основе. Этот завершающий этап
программы требует наибольших затрат материальных ресурсов и, следовательно,
наиболее тесной увязки с балансовыми расчетами в системе долгосрочного
планирования \cite{b5}. Заметим, что даже в том случае, когда схема управления
завершающим этапом программы еще не вышла из прогнозной стадии, описанная
методика ведения прогноза дает вероятностные оценки сроков и ресурсов,
необходимых для выполнения этого этапа, а также технологических нормативов
создаваемой в результате выполнения программы новой производственной базы. Иными
словами, имеется вся необходимая информация для проведения соответствующих
балансовых расчетов.
 
Помимо уже перечисленных проблем управления крупными целевыми программами и
управлением НТП, существует целый ряд задач меньшего масштаба, имеющих, тем не
менее, большое значение для ускорения темпов НТП. Одной из таких задач, которые
поставил В.М. Глушков, является коренное усовершенствование системы
научно-технической информации (НТИ). Действующая в СССР система управления была
направлена на то, чтобы оповещать широкий круг научно-технической общественности
о новых достижениях науки и техники. Нисколько не принижая важность этой задачи,
Глушков пишет, что ее необходимо было бы дополнить другой, не менее важной
задачей своевременного оповещения нужного круга лиц об актуальных нерешенных
проблемах НТП.
 
Такие задачи решались в Советском Союзе обычно бессистемно: предприятия, КБ и
НИИ, столкнувшись с проблемами, которые они сами решить не в силах, искали
возможных исполнителей сами, пользуясь собственной (как правило, весьма скудной)
информацией. В результате большинство проблем вообще не находило исполнителей (и
тем более самых лучших исполнителей), возникали неоправданные задержки и
дублирование в исполнении заданий. Усовершенствование системы НТИ, о котором
пишет Глушков, состоит в создании централизованной службы фиксации, оформления и
распределения между возможными исполнителями, возникающей научно-технической
проблематики.
 
Хочется отметить, что на сегодняшний день в Украине ни системы оповещения
научной общественности о новых достижениях науки, ни, тем более, оповещения об
актуальных нерешенных проблемах не существует. Т.е. можно констатировать
фактическое разрушение советской системы НТИ при полном отсутствии новой.
 
Глушков также придает большое значение проблеме достоверности информации, без
которой невозможно построить адекватной системы управления.
 
Еще одной проблемой Глушков считает заинтересованность предприятий во внедрении
научно-технических новинок. Полученная в результате НТП экономия должна идти на
расширение фондов стимулирования и для развития предприятий.
 
Также Глушков говорит об еще одном мощном рычаге ускорения НТП, а именно, об
автоматизации труда работников, осуществляющих этот прогресс. Это и
автоматизация экспериментальных исследований, и автоматизация НТИ, комплексная
автоматизация проектно-конструкторских работ и испытаний. Это не только
многократно позволяет ускорить НТП, но и улучшает качество получаемых
результатов.
 
Таким образом, до сих пор остаются неосуществленными планы по внедрению
комплексного подхода в управлении научно-техническим прогрессом. А без такого
управления общество обречено на стагнацию, лишается перспектив настоящего
развития и позитивных преобразований в интересах большинства. На наш взгляд,
применение разработок киевской школы кибернетики в новых технологических
условиях может изменить ситуацию даже на украинском уровне, направив оставшийся
научный потенциал для создания более гармоничного общества. При этом оптимальное
управление НТП должно быть сбалансированным и использовать все возможные рычаги.
 
\addcontentsline{toc}{section}{Список литературы}
\begin{thebibliography}{99}

 
\bibitem{b1}
Глушков В.М. Управление научно-техническим прогрессом. // Плановое хозяйство. –
1980. – №6. – с. 46-54

\bibitem{b2}
Глушков. В.М. О прогнозировании на основе экспертных оценок. // Кибернетика. –
1969. 2. – с. 2-4.

\bibitem{b3}
Прогнозирование научно-технического потенциала стран-членов СЭВ и инфраструктуры
НИОКР: концепция, разработки, опыт и перспективы. – Материалы к совещанию
экспертов и специалистов стран-членов СЭВ, 5-10 окт. 1987г., ЧССР. – Киев: ИЭС
им. Е.О. Патона АН УССР, 1987г. – 28с.

\bibitem{b4}
Глушков В.М. «Основы безбумажной информатики», М.: Наука, Гл. ред. физ.-мат.
лит., 1987. – 552 с.

\bibitem{b5}
Глушкова В., Карпець Э., ОГАС В. М. О возможности применения системы ДИСПЛАН для
сбалансированного управления экономикой. «Журнал соціальної критики Спільне:
COMMONS Journal of Social Criticism», 16 ноября 2016, [Электронный ресурс].

\end{thebibliography}

% Файл
% application/vnd.openxmlformats-officedocument.wordprocessingml.document iconupravlenie_nauchno_tehnicheskim_progressom_v.m.glushkova.docx (25.5 КБ)
% application/pdf iconcommons.com_.ua-upravlenie_nauchno-tehnicheskim_progressom_koncepciya_v_m_glushkova.pdf (97.02 КБ)
ОГАС
\url{ogas.kiev.ua} \copyright\ 2011
НТУУ «КПИ»Форум

\end{document}